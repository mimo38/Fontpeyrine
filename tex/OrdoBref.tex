\begin{multicols}{2}
\subsection*{Introït}
\rubrica{Pendant que le l'officiant se prépare à monter à l'autel avec les \emph{prières au bas de l'autel}, nous chantons l'Introït (voir à la messe du jour).}

\rubrica{On chante ensuite le \emph{Kyrie} et le \emph{Gloria}}

\subsection*{Oraison Collecte}
\rubrica{Cette oraison est appelée \emph{Collecte} parce que le prêtre y réunit, pour les offrir à Dieu, les prières et les vœux des fidèles assemblés. Elle termine généralement ainsi : «~Par Jésus-Christ Notre-Seigneur », pour nous montrer que nous ne pouvons aller au Père que par son divin Fils.}
\rubrica{Voir l'oraison à la messe du jour}

\subsection*{Épître}
\rubrica{Voir à la messe du jour. A la fin de l'épître, on répond~:}
\textbf{\rb\ Deo grátias.}

\subsection*{Graduel et Alléluia}
\rubrica{Graduel, Alléluia sont des chants de méditation sur les lectures ou le mystère liturgique de ce jour. \emph{Alleluia} signifie "louez Dieu". L'Alléluia se prolonge dans une mélodie ornée qui exprime la joie spirituelle au-delà des paroles.}
\rubrica{Voir à la messe du jour}

\subsection*{Évangile.}
\rubrica{On chante le saint Évangile tourné vers le Nord (le chœur étant normalement orienté vers l'Est) parce que le Nord symbolise les âmes qui se trouvent dans la froideur de l'infidélité et de l'ignorance du Christ, et qui sont éveillées par le souffle du Saint-Esprit.}
\rubrica{Voir à la messe du jour. A la fin de l'évangile, on répond~:}
\textbf{\rb\ Laus tibi Christe.}

\subsection*{Offertoire}
\rubrica{À l'Offertoire l'Église offre le Corps et le Sang du Christ, représentés par le pain et le vin. Notre Seigneur a communiqué à son Église le pouvoir d'offrir le même sacrifice qu'il offrit sur la Croix. L'Église s'y unit comme victime. Nous sommes donc une seule victime avec le Christ, unissant nos propres offrandes et nos sacrifices à celui du Christ et de l'Église. Notre participation à cette offrande est exprimée par le chant de l'offertoire.}
Recevez, Père saint, Dieu éternel et tout-puissant, cette offrande sans tache, que moi, votre indigne serviteur, je vous présente, à vous, mon Dieu vivant et vrai, pour mes péchés, offenses et négligences sans nombre, pour tous ceux qui m'entourent, ainsi que pour tous les fidèles vivants et morts~: qu'elle serve à mon salut et au leur pour la vie éternelle. Ainsi soit-il.

Nous vous offrons, Seigneur, le calice du salut, et nous demandons à votre clémence qu'il s'élève en parfum agréable devant votre divine Majesté, pour notre salut et celui du monde entier. \mbox{Ainsi soit-il}.\looseness=1
Priez, mes frères, pour que mon sacrifice, qui est aussi le vôtre, puisse être agréé par Dieu le Père tout-puissant.

\rubrica{À la messe lue on répond :}

℟. Que le Seigneur reçoive de vos mains le sacrifice, à la louange et à la gloire de son Nom, ainsi que pour notre bien et celui de toute sa sainte Église.

\subsection*{%
Oraison Secrète%
}

\rubrica{Voir à la messe du jour.}

\subsection*{Oblation du Sacrifice}

\rubrica{Commence ici la partie la plus sacrée de la Liturgie : le sacrifice réel et sacramentel où le prêtre immole le Christ. \emph{Vraiment à cette heure très redoutable il faut tenir haut son cœur vers Dieu, et non en bas vers la terre et les affaires terrestres. D'autorité donc le pontife  enjoint à cette heure à tous de laisser de côté les soucis, les sollicitudes domestiques, et de tenir leur cœur au ciel vers Dieu ami des hommes.} \mbox{(S. Cyrille de Jérusalem)}}

\subsection*{%
Préface}
\rubrica{Par le chant de la Préface le prêtre rend grâces à Dieu au nom de l'Église pour l'œuvre du salut réalisée par Jésus-Christ.}
\subsection*{%
Sanctus%
}

\rubrica{L'hymne au Dieu trois fois saint se compose d'une vision du prophète Isaïe (Is. 6,3) et des acclamations chantées lors de l'entrée triomphale du Christ à Jérusalem.}

\subsection*{Consécration}
\rubrica{Le prêtre récite alors l'histoire de l'institution de l'Eucharistie en accomplissant les mêmes gestes que le Christ.}
\end{multicols}

\greseparator{1}{5}
\versio{%
Qui prídie quam paterétur, accépit panem in sanctas ac venerábiles manus suas, et elevátis óculis in cælum ad te Deum Patrem suum omnipoténtem, tibi grátias agens, bene ✠ díxit, fregit, dedítque discípulis suis, dicens~: Accípite, et manducáte ex hoc omnes.}{%
Celui-ci, la veille de sa Passion, prit du pain dans ses mains saintes et adorables, et, les yeux levés au ciel vers vous, Dieu, son Père tout-puissant, vous rendant grâces, il ✠ bénit ce pain, le rompit et le donna à ses disciples en disant~: Prenez et mangez-en tous.}


\rubrica{Le prêtre prononce alors les paroles mêmes de Notre-Seigneur. Par ces paroles il opère la conversion du pain au saint Corps du Christ.}

\versio{%
\scalebox{.91}[1]{\textsc{\addfontfeatures{Renderer=Basic}Hoc est enim Corpus meum.}}\par%
}{%
\textsc{\addfontfeatures{Renderer=Basic}Car ceci est mon Corps.}\par%
}

%\vspace{1\baselineskip plus 2\baselineskip}

\rubrica{Ces paroles étant prononcées, le prêtre fait la génuflexion pour adorer le saint Corps, l'élève pour le présenter à l'adoration des fidèles, puis reprend le récit de l'institution de l'eucharistie~:}

\versio{%
Símili modo postquam cenátum est, accípiens et hunc præclárum Cálicem in sanctas ac venerábiles manus suas~: item tibi grátias agens, bene ✠ díxit, dedítque discípulis suis, dicens~: Accípite, et bíbite ex eo omnes.}{%
De même, après le repas, il prit aussi ce précieux calice dans ses mains saintes et adorables, vous rendit grâces encore, le ✠ bénit et le donna à ses disciples en disant~: Prenez et buvez-en tous.}

\rubrica{Prononçant alors les paroles mêmes de Notre-Seigneur, le prêtre opère la conversion du vin au précieux Sang du Christ.}
%
\nopagebreak\smallskip%
%
\versio{%
\textsc{\addfontfeatures{Renderer=Basic}Hic est enim Calix Sánguinis mei, novi et ætérni Testaménti~: mystérium fídei~: qui pro vobis et pro multis effundétur in remissiónem peccatórum.}%
}{%
\textsc{\addfontfeatures{Renderer=Basic}Car ceci est le Calice de mon Sang, le Sang de l'Alliance nouvelle et éternelle, le mystère de la foi, qui sera versé pour vous et pour un grand nombre en rémission des péchés.}%
}

\smallskip

\versio{%
Hæc quotiescúmque fecéritis, in mei memóriam faciétis.}{%
Toutes les fois que vous ferez cela, vous le ferez en mémoire de moi.}%

%\vspace{1\baselineskip plus 2\baselineskip}

\rubrica{Ces paroles étant prononcées, le prêtre fait la génuflexion pour adorer le précieux Sang. Il l'élève pour le présenter à l'adoration, et fait de nouveau la génuflexion.}

\greseparator{1}{5}

\begin{multicols}{2}

\subsection*{Communion}
\subsection*{%
Agnus Dei%
}

\rubrica{Les Hébreux immolaient un agneau en mémoire de leur délivrance de la captivité d'Égypte. Cet agneau était la figure du véritable Agneau immolé pour nous délivrer de la captivité du péché et du démon. Selon les paroles de S.~Jean-Baptiste, il est l'\emph{Agneau de Dieu qui enlève le péché du monde.}}

Agnus Dei, qui tollis peccáta mundi, miserére nobis. \emph{bis}%

{\textit Agneau de Dieu, qui ôtez les péchés du monde, ayez pitié de nous. \emph{bis}%
}%

{%
Agnus Dei, qui tollis peccáta mundi, dona nobis pacem.%
}{%
\textit Agneau de Dieu, qui ôtez les péchés du monde, donez-nous la paix.%
}%

Seigneur Jésus-Christ, qui avez dit à vos Apôtres~: C'est la paix que je vous laisse en héritage, c'est ma paix je vous donne, ne regardez pas mes péchés, mais la foi de votre Église~; daignez, selon votre volonté, lui donner la paix et la rassembler dans l'unité, vous qui, étant Dieu, vivez et régnez dans tous les siècles des siècles. Ainsi soit-il.

\rubrica{Le baiser de paix, manifestant l'union dans la paix du Christ, est hiérarchiquement transmis de l'autel jusqu'au dernier degré du clergé. \emph{Ne pense pas que ce baiser soit comme ceux qui se donnent sur la place entre amis ordinaires. Il unit les âmes entre elles~; il est réconciliation, et pour cette raison il est saint.} (S.~Cyrille de Jérusalem)}

Seigneur Jésus-Christ, Fils du Dieu vivant, qui, accomplissant la volonté du Père dans une œuvre commune avec le Saint-Esprit, avez par votre mort donné la vie au monde, délivrez-moi par votre Corps et votre Sang infiniment saints de tous mes péchés et de tout mal. Faites que je reste toujours attaché à vos commandements, et ne permettez pas que je sois jamais séparé de vous, qui, étant Dieu, vivez et régnez avec Dieu le Père et le Saint-Esprit dans les siècles des siècles. Ainsi soit-il.

Seigneur Jésus-Christ, si j'ose recevoir votre Corps malgré mon indignité, que cela n'entraîne pour moi ni jugement ni condamnation, mais, par votre miséricorde, me serve de sauvegarde et de remède pour l'âme et pour le corps, vous qui, étant Dieu, vivez et régnez avec Dieu le Père en l'unité du Saint-Esprit, dans tous les siècles des siècles. Ainsi soit-il.
\end{multicols}

\greseparator{1}{5}

\rubrica{Après avoir communié, le prêtre se retourne vers les fidèles en leur présentant le saint Corps du Christ.}

\versio{%
Ecce Agnus Dei, ecce qui tollis peccáta mundi.}{%
Voici l'Agneau de Dieu, voici celui qui enlève les péchés du monde.}

\rubrica{Nous répondons par les paroles du bienheureux centurion. Cette fois ce n'est plus la guérison d'un serviteur que nous demandons, mais celle de notre âme.}
%
\reponsegras{%
Dómine, non sum dignus, ut intres sub tectum meum~: sed tantum dic verbo, et sanábitur ánima mea. \emph{ter}}{%
Seigneur, je ne suis pas digne que vous entriez sous mon toit~; mais dites seulement une parole, et mon âme sera guérie. \emph{ter}}

\begin{framed}
\rubrica{Aux premiers siècles le diacre invitait ceux qui ne pouvaient pas communier à se retirer. Aujourd'hui l'Église est plus indulgente mais les conditions pour communier sont les mêmes~:}

\begin{itemize}
\item être baptisé et catholique~; 
\item ne pas avoir de péché mortel sur la conscience, ce qui suppose notamment l'observation des lois de l'Église au sujet du mariage~; 
\item avoir observé le jeûne eucharistique. La discipline actuelle est d'une heure de jeûne avant la communion (nous conseillons cependant de s'en tenir à la discipline d'avant le Concile Vatican II~: trois heures pour la nourriture solide, une heure pour le liquide non alcoolisé).
\end{itemize}

\rubrica{La sainte communion est reçue à genoux et directement dans la bouche.}
\end{framed}

\needspace{3\baselineskip}
\rubrica{En donnant la sainte Eucharistie, le prêtre dit~:}

\versio{%
Corpus Dómini nostri Iesu Christi custódiat ánimam tuam in vitam ætérnam. Amen.}{%
Que le Corps de Notre Seigneur Jésus-Christ garde votre âme pour la vie éternelle. Ainsi soit-il.}

\rubrica{%
On ne répond rien.}
\greseparator{1}{5}

\begin{multicols}{2}

\rubrica{%
Jésus-Christ est le Bon Pasteur qui donne sa vie pour ses brebis. \emph{Je suis venu pour qu'ils aient la vie et qu'ils l'aient en abondance}. Il nous communique cette vie dans la sainte Eucharistie.%
}

\subsection*{Chant de communion et Postcommunion}
\rubrica{Voir à la messe du jour}
\end{multicols}