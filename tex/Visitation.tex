\addchap{Messe de la Visitation - 2 juillet}

\rubrica{ Le 2 juillet 1769, fête de la Visitation, un orage, tel qu'on n'en avait pas vu depuis longtemps, dévasta les campagnes de Saint-Cyprien, du Bugue, et de Terrasson. La paroisse de Tursac et le sanctuaire de Fontpeyrine furent seuls épargnés.
C'est pourquoi, les représentants de la ville de Tursac et sa pieuse population firent vœu d'aller au sanctuaire tous les ans en procession, au jour anniversaire du bienfait, pour remercier Notre-Dame de Fontpeyrine.}

\rubrica{La messe est à la page \emph{\pageref{Ordo}}, excepté les textes propres qui suivent~:}

\begin{multicols}{2}
\subsection*{Introït}
{Salut, ô Mère sainte~; mère qui avez enfanté le Roi qui régit le ciel et la terre dans les siècles des siècles.}
De mon cœur a jailli une parole excellente, c’est que je consacre mes œuvres à mon Roi. \vb\ Gloire au Père.

\subsection*{Collecte}
{Prions\\
Seigneur, nous vous prions d’accorder à vos serviteurs le don de la grâce céleste~: et, comme l’enfantement de la bienheureuse Vierge a été le principe de leur salut~; qu’ainsi la pieuse solennité de sa Visitation leur procure un accroissement de paix. Par Notre Seigneur.}
{\textbf{ \rb\ Amen.}}


\subsection*{Épître : Lecture du Livre de la Sagesse}
\rubrica{Cant. 2, 8-14.}
Voici qu’il vient, bondissant sur les montagnes, sautant sur les collines. Mon bien-aimé est semblable à la gazelle, ou au faon des biches. Le voici, il est derrière notre mur, regardant par la fenêtre, épiant par le treillis. Voici, mon bien-aimé me dit~: "Lève-toi, hâte-toi, mon amie, ma colombe, ma belle, et viens ! Car voici que l’hiver est fini~; la pluie a cessé, elle a disparu. Les fleurs ont paru sur notre terre, le temps des chants est arrivé~; la voix de la tourterelle s’est fait entendre dans nos campagnes~; le figuier pousse ses fruits naissants, les vignes en fleur donnent son parfum. Lève-toi, mon amie, ma belle, et viens ! Ma colombe, qui te tiens dans la fente du rocher, dans l’abri des parois escarpées, montre-moi ton visage, que ta voix résonne à mes oreilles~; car ta voix est douce, et ton visage charmant.
\textbf{\rb\ Deo grátias.}

\subsection*{Antiennes}
Vous êtes bénie et digne de vénération, Vierge Marie, qui avez été mère du Sauveur, sans que votre pureté ait subi d’atteinte. Vierge, Mère de Dieu, celui que tout l’univers ne peut contenir, s’est enfermé dans votre sein en se faisant homme.

Alléluia, alléluia. \vb\ Vous êtes heureuse, sainte Vierge Marie, et tout à fait digne de louange, car de vous est sorti le soleil de justice, le Christ notre Dieu. Alléluia.

\subsection*{Lecture du saint Évangile selon saint Luc.}
\rubrica{Luc 1, 39-47.}
En ces jours-là, Marie partit et s’en alla en hâte vers la montagne, en une ville de Juda. Et elle entra dans la maison de Zacharie, et salua Élisabeth. Or, quand Élisabeth entendit la salutation de Marie, l’enfant tressaillit dans son sein, et elle fut remplie du Saint-Esprit. Et elle s’écria à haute voix, disant~: "Vous êtes bénie entre les femmes, et le fruit de vos entrailles est béni. Et d’où m’est-il donné que la mère de mon Seigneur vienne à moi ? Car votre voix, lorsque vous m’avez saluée, n’a pas plus tôt frappé mes oreilles, que l’enfant a tressailli de joie dans mon sein. Heureuse celle qui a cru ! Car elles seront accomplies les choses qui lui ont été dites de la part du Seigneur !" Et Marie dit~: "Mon âme glorifie le Seigneur, et mon esprit tressaille de joie en Dieu, mon Sauveur".
{\textbf {\rb\ Laus tibi Christe.}}

\rubrica{Chant du Credo page \pageref{Credo}}

\subsection*{Antienne d'Offertoire}
Vous êtes bienheureuse, Vierge Marie, qui avez porté le Créateur de toutes choses ; vous avez enfanté celui qui vous a créée, et vous demeurez à jamais Vierge.

\subsection*{Oraison Secrète}
Qu’elle nous porte secours, Seigneur, la bonté de votre Fils unique, qui né d’une Vierge, n’a point altéré l’intégrité de sa Mère mais l’a consacrée, afin que nous purifiant de nos fautes en la solennité de sa Visitation, il vous rende notre oblation agréable, lui Jésus-Christ notre Seigneur.

\subsection*{Chant de communion}
Bienheureux le sein de la Vierge Marie, qui a porté le Fils du Père éternel.

\needspace{3\baselineskip}
\subsection*{Postcommunion}
{Nous avons reçu, Seigneur, les choses saintes qui vous sont offertes en cette solennité annuelle, faites, nous vous en supplions, qu’elles nous donnent les remèdes spirituels utiles à la vie temporelle et conduisant à la vie éternelle.}
\textbf{ \rb\ Amen.}

\end{multicols}