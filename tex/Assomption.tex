\addchap{Messe de l'Assomption - 15 août}

%\rubrica{ Le 2 juillet 1769, fête de la Visitation, un orage, tel qu'on n'en avait pas vu depuis longtemps, dévasta les campagnes de Saint-Cyprien, du Bugue, et de Terrasson. La paroisse de Tursac et le sanctuaire de Fontpeyrine furent seuls épargnés. C'est pourquoi, les représentants de la ville de Tursac et sa pieuse population firent vœu d'aller au sanctuaire tous les ans en procession, au jour anniversaire du bienfait, pour remercier Notre-Dame de Fontpeyrine.}

\rubrica{La messe est à la page \emph{\pageref{Ordo}} excepté les textes propres qui suivent~:}

\begin{multicols}{2}
\subsection*{Introït}
{Il parut dans le ciel un grand signe : une femme revêtue du soleil, la lune sous ses pieds, et une couronne de douze étoiles sur sa tête.
Chantez au Seigneur un cantique nouveau : car il a fait des merveilles.
\vb\ Gloire au Père.

\subsection*{Collecte}
Prions\\
Dieu éternel et tout-puissant, vous avez élevé, en son corps et en son âme, à la gloire du ciel, Marie, la Vierge immaculée, mère de votre Fils : faites, nous vous en prions, que, sans cesse tendus vers les choses d’en-haut, nous méritions d’avoir part à sa gloire.
{\textbf \rb\ Amen.}

\subsection*{Epître : Lecture du Livre de Judith.}
\rubrica{Iudith. 13, 22-25 ; 15, 10.}
Le Seigneur t’a bénie dans sa force, car par toi il a réduit à néant tous nos ennemis. Ma fille, tu es bénie par le Seigneur, le Dieu très haut, plus que toutes les femmes qui sont sur la terre. Béni soit le Seigneur, créateur du ciel et de la terre, qui a conduit ta main pour trancher la tête au plus grand de nos ennemis ! Il a rendu aujourd’hui ton nom si glorieux, que ta louange ne disparaîtra pas de la bouche des hommes, qui se souviendront éternellement de la puissance du Seigneur ; car, en leur faveur, tu n’as pas épargné ta vie en voyant les souffrances et la détresse de ta race, mais tu nous as sauvés de la ruine en marchant dans la droiture en présence de notre Dieu. Tu es la gloire de Jérusalem ; tu es la joie d’Israël ; tu es l’honneur de notre peuple.
\textbf{\rb\ Deo grátias.}

\subsection*{Antiennes}
Écoutez, ma Fille, voyez et tendez l’oreille : le Roi désirera votre beauté.
V/. Toute belle s’avance la fille du Roi, son vêtement est fait de tissus d’or.

Allelúia, allelúia. \vb\ Marie a été élevée dans les Cieux : l’armée des Anges se réjouit. Alléluia.


\subsection*{Lecture du Saint Évangile selon saint Luc.}
\rubrica{Luc. 1, 41-50.}
En ce temps-là : Élisabeth fut remplie du Saint-Esprit et elle s’écria à haute voix, disant : « Vous êtes bénie entre les femmes, et le fruit de vos entrailles est béni. Et d’où m’est-il donné que la mère de mon Seigneur vienne à moi ? Car votre voix, lorsque vous m’avez saluée, n’a pas plus tôt frappé mes oreilles, que l’enfant a tressailli de joie dans mon sein. Heureuse vous qui avez cru ! Car elles seront accomplies les choses qui vous ont été dites de la part du Seigneur ! ». Et Marie dit : « Mon âme glorifie le Seigneur, et mon esprit tressaille de joie en Dieu, mon Sauveur, parce qu’il a jeté les yeux sur la bassesse de sa servante. Voici, en effet, que désormais toutes les générations me diront bienheureuse, parce que le Puissant a fait pour moi de grandes choses. Et son nom est saint, et sa miséricorde d’âge en âge pour ceux qui le craignent ».
{\textbf \rb\ Laus tibi Christe.}

\subsection*{Antienne d'offertoire}
Je mettrai une hostilité entre toi et la femme, entre ton lignage et le sien.

\subsection*{Oraison Secrète}
Que cette offrande de notre dévotion monte jusqu’à vous, Seigneur, et que par l’intercession de la très bienheureuse Vierge Marie, élevée au ciel, nos cœurs, brûlants du feu de la charité, soient constamment tendus vers vous.

\subsection*{Chant de communion}
Toutes les générations me diront bienheureuse, parce que le Puissant a fait pour moi de grandes choses.

\subsection*{Postcommunion}
Ayant reçu le sacrement du salut, nous supplions, Seigneur, par les mérites et l’intercession de la bienheureuse Marie, élevée au ciel, de nous faire parvenir à la gloire de la résurrection.
{\textbf \rb\ Amen.}
\end{multicols}