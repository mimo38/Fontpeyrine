\addsec{Prières pour l'action de grâces après la communion}

\subsection{Âme de Jésus-Christ}

Âme de Jésus-Christ, sanctifiez-moi. Corps de Jésus-Christ, sauvez-moi. Sang de Jésus-Christ, enivrez-moi. Eau du côté de Jésus-Christ, purifiez-moi. 

Passion de Jésus-Christ, fortifiez-moi. Ô bon Jésus, exaucez-moi. Dans Vos saintes plaies, cachez-moi. Ne permettez pas, que je me sépare de Vous. 

De l’esprit malin, défendez-moi. À l’heure de ma mort, appelez-moi. Et commandez, que je vienne à Vous, Afin qu’avec Vos Saints, je Vous loue Dans les siècles des siècles. Ainsi soit-il. 


\subsection{O bon et très doux Jésus}
O bon et très doux Jésus,
je me prosterne à genoux
en votre présence.

Je vous prie
et je vous conjure,
avec toute la ferveur de mon âme,
de daigner graver dans mon coeur
de vifs sentiments de foi,
d'espérance et de charité;
un vrai repentir de mes péchés
et une volonté très ferme
de m'en corriger,
pendant que je considère
et contemple en esprit vos cinq plaies,
avec une grande affliction
et une grande douleur,
ayant devant les yeux
ces paroles que le prophète David
mettait dans votre bouche,
ô bon Jésus:
"Ils ont percé mes mains et mes pieds:
ils ont compté tous mes os."

\subsection{Souvenez-vous}
Souvenez-vous, ô très miséricordieuse Vierge Marie, qu’on n’a jamais entendu dire qu’aucun de ceux qui ont eu recours à votre protection, imploré votre assistance ou réclamé votre intercession, ait été abandonné de vous.

Animé d’une pareille confiance, je cours vers vous, ô Vierge des vierges et notre Mère, je viens à vous, et, gémissant sous le poids de mes péchés, je me prosterne à vos pieds.

Ô Mère du Verbe, ne rejetez pas mes prières, mais écoutez-les favorablement et daignez les exaucer. Ainsi soit-il.

\needspace{8\baselineskip}
\subsection{Sub tuum præsidium}
\versio{%
 Sub tuum praesidium confugimus, sancta Dei Genitrix : nostras deprecationes ne despicias in necessitatibus, sed a periculis cunctis libera nos semper, Virgo gloriosa et benedicta.}%
 {Sous votre protection nous venons nous réfugier, sainte Mère de Dieu~;ne rejetez pas les prières que nous vous adressons dans nos pressants besoins ; mais délivrez-nous toujours de tous les dangers, Vierge glorieuse et bénie.}