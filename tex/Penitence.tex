\newcommand{\commandement}[1]{\noindent\textbf{#1}}
\addchap{Sacrement de Pénitence}

Le sacrement de Pénitence ou confession fut institué par Notre Seigneur Jésus-Christ, pour effacer les péchés commis après le baptême. Notre-Seigneur l’a transmis aux Apôtres lorsqu’il leur a dit : « Recevez le Saint-Esprit ; les péchés seront remis à ceux à qui vous les remettrez et ils seront retenus à ceux à qui vous les retiendrez. »


\addsec[Pour les enfants]{\underline{Pour les enfants}}

Pour bien recevoir le sacrement de Pénitence, il faut connaître ses péchés, en avoir la contrition, les accuser, et après en avoir reçu l’absolution, faire la pénitence imposée par le prêtre.

\subsection*{Qu’est-ce que se confesser ?}

C’est dire ses péchés à un prêtre pour en recevoir l’absolution.

\subsection*{Comment se confesser ?}

Deux choses à faire :
\begin{itemize}
\item retrouver nos péchés ;
\item les regretter.
\end{itemize}

\emph{Pour les chercher}, fermons les yeux et faisons l’examen de conscience. Nous pouvons nous aider de l’examen de conscience qui suit.
\pagebreak[3]

\emph{Pour les regretter}, disons lentement :\\
Esprit-Saint, qui êtes la lumière de nos cœurs, éclairez ma conscience, montrez-moi mes péchés, faites que je les voie, comme je les verrai à l’heure de mon jugement et comme les voyait Jésus, quand il mourait pour les réparer. Montrez-moi les défauts qui m’ont poussé à les commettre, pour que je les combatte. Faites que je sois bien décidé à suivre les conseils de votre prêtre, pour que votre grâce en moi rencontre moins d’obstacles, et puisse me guérir de mes mauvais penchants. Ainsi-soit-il !

\subsection*{Examen de conscience}

\minisec{Confession précédente}

\begin{itemize}
\item Combien y a-t-il de temps que je ne me suis pas confessé ?
\item Ai-je bien dit tous mes péchés ?
  \begin{itemize}
  \item N’ai-je pas caché volontairement des péchés graves ? (Si oui, il vous faut absolument vous en accuser, car non seulement aucun de vos péchés accusés n’a été pardonné, mais vous avez ajouté un autre péché très grave : un sacrilège).
  \item N’ai-je pas oublié des péchés graves ? (Si oui, votre dernière confession a été bonne quand même ; mais il faut que vous les accusiez maintenant).
  \end{itemize}
\item Me suis-je mal préparé à ma dernière confession ?
\item N’ai-je pas manqué de contrition, c’est-à-dire de vrai repentir de mes fautes ? (pour avoir un vrai repentir, il faut être décidé à faire tout son possible pour ne pas recommencer).
\item Ai-je fait ma pénitence ?
\item Quelle résolution avais-je prise lors de ma dernière confession ? Est-ce que je l’ai tenue ?
\end{itemize}

\needspace{3\baselineskip}
\minisec{Commandements de Dieu}

\commandement{1. Tu adoreras Dieu seul et tu l'aimeras plus que tout.}

\begin{itemize}
\item Ai-je manqué mes prières ? du matin ? du soir ? les ai-je mal faites ?
\item Me suis-je mal tenu à l’église ? Ai-je dissipé les autres ?
\item Ai-je eu honte de paraître chrétien ?
\item Ai-je tenu des conversations contre la religion ?
\end{itemize}

\commandement{2. Tu ne prononceras le nom de Dieu qu’avec respect.}
\begin{itemize}
\item Ai-je dit des gros mots ? Ai-je dit des jurons ?
\item Ai-je fait des serments pour des riens ?
\end{itemize}

\commandement{3. Tu sanctifieras le jour du Seigneur.}
\begin{itemize}
\item Ai-je manqué par ma faute, la messe le dimanche ou les fêtes d’obligation ? Combien de fois ?
\item Suis-je arrivé en retard ? À quel moment ?
\end{itemize}

\commandement{4. Tu honoreras ton père et ta mère.}
\begin{itemize}
\item Ai-je désobéi à mes parents ?
\item Leur ai-je mal répondu ? Me suis-je moqué d’eux ?
\item Ai-je fait la tête ? Ai-je fait du mauvais esprit ?
\end{itemize}

\commandement{5. Tu ne tueras pas.}
\begin{itemize}
\item Me suis-je disputé avec les autres ?
\item Ai-je gardé rancune ? Ai-je cherché à me venger ?
\item Ai-je donné le mauvais exemple ? ou entraîné d’autres à pécher ?
\end{itemize}

\commandement{6. Tu ne feras pas d'impureté.\\
9. Tu n'auras pas de désir impur volontaire.}
\begin{itemize}
\item Ai-je regardé des images mauvaises, impures ? Ai-je cherché exprès des journaux impurs ?
\item Ai-je vu des spectacles mauvais (à la télévision par exemple) ?
\item Ai-je accepté des pensées impures ? des désirs impurs ?
\item Ai-je participé à de mauvaises conversations ?
\item Ai-je fait des actions impures ? seul ? avec d’autres ?
\end{itemize}

\commandement{7. Tu ne voleras pas.\\
10. Tu ne désireras pas injustement le bien des autres.}
\begin{itemize}
\item Ai-je pris ou recherché à prendre quelque chose qui n’était pas à moi (des gourmandises, de l’argent) ?
\item Ai-je abîmé exprès ce qui ne m’appartenait pas ?
\item Ai-je triché au jeu ? Ai-je copié en classe, à un examen, à une composition, à un devoir ?
\end{itemize}

\commandement{8. Tu ne mentiras pas.}
\begin{itemize}
\item Ai-je menti ? (pour m’amuser, pour me vanter, pour ne pas être puni, pour tromper).
\item Ai-je dit du mal des autres ? Ai-je cherché à faire punir les autres ?
\item  Ai-je pensé sans raison suffisante, du mal des autres ?
\end{itemize}


\needspace{3\baselineskip}
\minisec{Commandements de l'Église}

\begin{itemize}
\item Me suis-je bien préparé à ma dernière
 communion ?
\item Ai-je communié sans être à jeun ?
\item Ai-je communié avec des péchés graves sur la conscience ?
\item Ai-je mangé de la viande les jours défendus ?
\end{itemize}


\needspace{3\baselineskip}
\minisec{Péchés capitaux}

\begin{itemize}
\item Ai-je été orgueilleux ? Ai-je refusé de reconnaître mes torts ? Ai-je rabaissé les autres en pensée ? Me suis-je vanté ? Me suis-je vexé pour rien ?
\item Ai-je été gourmand : en étant difficile ? en mangeant trop de friandises ? en
mangeant et en buvant avec excès ? Ai-je fumé en cachette ?
\item Ai-je été avare ? Ai-je refusé de prêter mes affaires ?
\item Ai-je été jaloux ?
\item Me suis-je mis en colère ? Ai-je eu mauvais caractère, rendant la vie pénible
autour de moi ? Ai-je été impatient ? Ai-je fait mettre exprès les autres en colère ?
\item Ai-je été paresseux ? pour me lever ? pour prier ? pour communier ? à l’école ? au catéchisme ? pour faire mes devoirs ? Ai-je manqué par ma faute l’école ou le catéchisme ?
\end{itemize}

Mon défaut dominant est : …

Je prends la résolution de : …

\clearpage
\addsec[Pour les adultes]{\underline{Sacrement de pénitence pour les adultes}}

Le sacrement de Pénitence, appelé aussi confession, est le sacrement institué par Jésus-Christ pour remettre les péchés commis après le baptême. Le ministre de ce sacrement est le prêtre. Tenant la place de Jésus-Christ et recevant la confidence de nos péchés même les plus cachés, le prêtre est tenu à un secret absolu sur tout ce qu’il a entendu. C’est le secret de confession. Même sous la menace de mort ou de torture, il ne peut rien dire et rien révéler. Nous pouvons donc lui parler en toute confiance et sans crainte.

Ce sacrement exige quatre conditions :
\begin{enumerate}
\item la connaissance de nos péchés ;
\item la contrition de nos péchés ;
\item la confession de nos péchés au prêtre suivie de l’absolution ;
\item la satisfaction pour nos péchés.
\end{enumerate}

\smallskip
Les parties du sacrement sont :
\begin{itemize}
\item la contrition : c’est un acte de volonté, une douleur de l’âme et l’horreur du péché commis, et la résolution de ne plus pécher à l’avenir ;
\item la confession : elle consiste dans l’accusation détaillée de nos péchés faite au confesseur pour en avoir l’absolution et la pénitence ;
\item l'absolution : c’est la phrase que le prêtre prononce au nom de Jésus-Christ, pour remettre les péchés au pénitent ;
\item la satisfaction : ou pénitence sacramentelle, c’est la prière ou la bonne œuvre imposée par le confesseur pour le châtiment et la correction du pêcheur, et l’escompte de la peine temporelle méritée en péchant.
\end{itemize}

\smallskip
\textbf{Les effets de la confession bien faite} : le sacrement de Pénitence :
\begin{itemize}
\item donne la grâce sanctifiante avec laquelle les péchés mortels, et aussi les péchés véniels, confessés et que l'on regrette, nous sont remis ;
\item commue la peine éternelle en temporelle ; celle-ci est diminuée dans la mesure de la contrition ;
\item rend les mérites des bonnes œuvres faites avant de commettre le péché
mortel ;
\item donne à l’âme les secours nécessaires pour ne pas retomber dans le péché et redonner la paix à la conscience.
\end{itemize}

\smallskip
\textbf{Pour préparer une bonne confession :}

dans la confession il faut accuser au moins tous les péchés mortels, pas encore bien confessés (dans une bonne confession) et ceux que l’on se rappelle. Indiquer, dans la mesure du possible, leur espèce et leur nombre.

Pour cela on demande à Dieu la grâce de bien connaître ses fautes, et on s’examine sur les dix commandements et les préceptes de l’Église, sur les péchés capitaux et les devoirs de notre état.

N.B. 1. Pour reconnaître un péché mortel (c’est-à-dire qui donne la mort surnaturelle à l’âme), il faut trois choses :
\begin{itemize}
\item la gravité de la matière ;
\item la pleine advertance (c’est-à-dire la pleine connaissance) ;
\item le plein consentement.
\end{itemize}

2. L’accusation de l’espèce et du nombre est de rigueur pour les désirs, au moins approximativement s'il n'est pas possible de se souvenir du nombre exact.


\subsection*{Examen de conscience}

\minisec{Commandements de Dieu}

\commandement{1. Tu adoreras Dieu seul et tu l'aimeras plus que tout.}

Manqué à mes prières, les ai mal faites.
Craint de me montrer chrétien, par respect humain. Négligé de m’instruire des
vérités de la religion, doutes volontaires.
Lu des livres, des journaux impies. Parlé,
agi contre la religion. Murmuré contre
Dieu et sa Providence. Appartenu à des
sociétés impies (franc-maçonnerie, communisme, sectes hérétiques, etc.) Pratiqué
des superstitions, consulté les cartes et les
devins. Avoir tenté Dieu.

Péchés contre la foi : refuser d’admettre
une ou plusieurs vérités révélées de Dieu.
Péchés contre l’espérance : manquer de
confiance en la bonté et Providence de
Dieu. Prétendre qu’il soit impossible de
vivre en vrai chrétien quoiqu’on en demande la grâce. Pécher par présomption
en abusant de la bonté de Dieu.

Péchés contre la charité : refuser d’aimer
Dieu par-dessus tout. Passer des semaines
et des mois sans faire le plus petit acte
d’amour de Dieu. Indifférence religieuse.
Sacrilèges en profanant les choses saintes,
en particulier confessions et communions
sacrilèges.

Charité envers le prochain : refuser de
voir Dieu dans nos frères, d’aimer Dieu
dans le prochain. Mépriser, détester, se
moquer du prochain.

\commandement{2. Tu ne prononceras le nom de Dieu qu’avec respect.}

Fait des serments faux ou inutiles — Imprécations contre moi-même ou contre
d’autres — Manqué de respect à l’égard
du nom de Dieu ou des saints — Blasphémé en murmurant contre Dieu dans
les épreuves — Manqué à des vœux.

\commandement{3. Tu sanctifieras le jour du Seigneur.}

À ce commandement se rapportent les 1\ier\ et 2\ieme\ commandements de l’Église.
Manqué à la messe le dimanche par ma faute, arrivé
en retard, assisté sans respect. Travaillé
ou fait travailler sans nécessité et sans
permission. Avoir profané
cette journée par des réunions ou amusements dangereux pour la foi ou les
mœurs.

\commandement{4. Tu honoreras ton père et ta mère.}

Enfants : manqué de respect. Désobéi.
Causé du chagrin à mes parents. Négligé
de les assister. N’avoir pas tenu compte de
leurs sages avis.

Parents : ai-je pensé à donner ou procurer
une instruction religieuse à nos enfants ?
Les ai-je fait prier ? Ai-je choisi pour eux
l’école la plus sûre ? Veillé sur eux avec
diligence ? Les ai-je conseillés, repris,
corrigés ?

Époux : l’amour entre les conjoints est-il
vraiment patient, empressé, prêt à tout ?
Manque de support mutuel. Avoir critiqué
son conjoint devant les enfants.

Inférieurs (employés, ouvriers, soldats) :
manque de respect, d’obéissance à mes
supérieurs — Fait du tort par des critiques
— Négligé mon service. Commis des abus
de confiance.

Supérieurs : manqué à la justice commutative, en ne donnant pas ce qui était dû, à
la justice sociale. Manqué à la charité, en
ne procurant pas les secours nécessaires
— N’avoir pas traité ses employés avec
bonté, équité, charité.

\commandement{5. Tu ne tueras pas.}

M’être mis en colère. Voulu me venger.
Souhaité du mal — Haines, rancunes, refuser de pardonner — Ai injurié, blessé
— Impatiences ­ — Mauvais conseils —
Scandalisé par paroles... actions — Infanticide — Avortement — Euthanasie —
Transgressions graves au code de la route,
de façon volontaire (même s’il n’est rien
arrivé).

\commandement{6. Tu ne feras pas d'impureté.\\
9. Tu n'auras pas de désir impur volontaire.}

M’être arrêté volontairement à des pensées, à des désirs contraires à la pureté
— Conversations et chansons légères ou
déshonnêtes, vêtements indécents — Télévision, radio (mauvaises émissions),
gravures, livres, journaux mauvais — Regards, familiarités coupables — Actions
déshonnêtes, seul... avec d’autres —
Liaisons ou fréquentations coupables —
Fraudes dans l’usage du mariage — Refus
du dû conjugal — Péchés entre fiancés.

\commandement{7. Tu ne voleras pas.\\
10. Tu ne désireras pas injustement le bien des autres.}

Désiré prendre le bien d’autrui ­ — Commis ou aidé à commettre des injustices,
des fraudes, des vols — Causé du dommage — Pas restitué — Pas payé mes dettes — Fait tort dans les ventes, contrats,
transactions, etc.

\commandement{8. Tu ne mentiras pas.}

Ai menti — Jugé témérairement — Dit
du mal du prochain — Calomnié — Faux
témoignage — Violé des secrets (lu une
lettre). Fait ou diffusé des soupçons.


\minisec{Commandements de l'Église}

\commandement{1. Les fêtes tu sanctifieras qui te sont de commandement.\\
2. Les dimanches Messe tu entendras, et les fêtes pareillement.}

Voir au 3\ieme\ commandement de Dieu.

\commandement{3. Tous tes péchés confesseras, à tout le moins une fois l’an.\\
4. Ton Créateur tu recevras, au moins à Pâques humblement.}

Être resté plus d'un an sans confesser un péché grave. Ne pas avoir communié, au moins à Pâques. Avoir fait des communions sacrilèges.

\commandement{5. Vigiles, pénitence feras − Carême et Quatre-Temps également.\\
6. De la viande ne mangeras les jours défendus mêmement.}

Avoir sans raison légitime et sans permission : manqué au jeûne - mangé de la viande les jours défendus - manqué au
jeûne eucharistique.

\commandement{7. Loi de l'Index}

Avoir lu ou conservé des livres, revues ou journaux défendus expressément par l’Église ou contre la foi et les mœurs.


\minisec{Péchés capitaux}

\commandement{Orgueil}

Agi par orgueil - Dépenses et luxes exagérés - Méprisé les autres - M’être complu
dans des pensées de vanité - Susceptibilités - Être esclave du « qu’en dira-t-on ? »
et la mode.

\commandement{Avarice}

M’être trop attaché à l’argent - N’avoir
pas fait l’aumône selon mes moyens - Jeux
d’argent (voir aux 7\ieme\ et 10\ieme\ commandements
de Dieu).

\commandement{Luxure} : voir aux 6\ieme\ et 9\ieme\ commandements de Dieu.

\commandement{Envie}

Avoir entretenu des sentiments de jalousie - Cherché à nuire aux autres par envie
- M’être réjoui du mal ou attristé du bien
d’autrui.

\commandement{Gourmandise}

Excès dans le manger, dans le boire.
Ivresse : combien de fois ? Usage de stupéfiants.

\commandement{Colère} : voir au 5\ieme\ commandement de Dieu.

\commandement{Paresse}

Au lever. Dans le travail. Dans les devoirs religieux.
