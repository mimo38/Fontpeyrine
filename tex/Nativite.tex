\addchap{Messe de la Nativité de la Vierge - 8 septembre}

\rubrica{Avant la proclamation dogme de l'Immaculé Conception (1854) et sa fête le 8 décembre, la fête de la nativité de la Vierge était celle qui commémorait cette vérité si importante : que le péché n'a pas atteint Marie, même pas le péché originel. Elle naît donc immaculée, et voilà pourquoi nous fêtons sa naissance.}

\rubrica{À Fontpeyrine, comme de nombreux sanctuaires périgourdins et français (Capelou, Laveyssière, Sanilhac,…), c'est aujourdhui la fête patronale.}

\rubrica{La messe est à la page \emph{\pageref{Ordo}} excepté les textes propres qui suivent~:}

\begin{multicols}{2}
\subsection*{Introït}
Salut, ô Mère sainte ; mère qui avez enfanté le Roi qui régit le ciel et la terre dans les siècles des siècles.
De mon cœur a jailli une parole excellente, c’est que je consacre mes œuvres à mon Roi.
\vb\ Gloire au Père.

\subsection*{Collecte}
Prions\\
Seigneur, nous vous prions d’accorder à vos serviteurs le don de la grâce céleste ; et, comme l’enfantement de la bienheureuse Vierge a été le principe de leur salut, qu’ainsi la pieuse solennité de sa Nativité leur procure un accroissement de paix.
{\textbf \rb\ Amen.}

\subsection*{Epître : Lecture du Livre de Judith.}
\rubrica{Prov. 8, 22-35.}
Le Seigneur m’a possédée au commencement de ses voies, avant de faire quoi que ce soit, dès le principe. J’ai été établie dès l’éternité, et dès les temps anciens, avant que la terre fût créée. Les abîmes n’étaient pas encore, et déjà j’étais conçue ; les sources des eaux n’avaient pas encore jailli ; les montagnes ne s’étaient pas encore dressées avec leur pesante masse ; j’étais enfantée avant les collines. Il n’avait pas encore fait la terre, ni les fleuves, ni les bases du globe terrestre. Lorsqu’il préparait les cieux, j’étais là ; lorsqu’il environnait les abîmes de leurs bornes, par une loi inviolable ; lorsqu’il affermissait l’air dans les régions supérieures, et qu’il équilibrait les sources des eaux ; lorsqu’il entourait la mer de ses limites, et qu’il imposait une loi aux eaux, pour qu’elles ne franchissent point leurs bornes, lorsqu’il posait les fondements de la terre, j’étais avec lui, réglant toutes choses, et j’étais chaque jour dans les délices, me jouant sans cesse devant lui, me jouant sur le globe de la terre, et mes délices sont d’être avec les enfants des hommes. Maintenant donc, mes fils, écoutez-moi : Heureux ceux qui gardent mes voies. Ecoutez mes instructions et soyez sages, et ne les rejetez pas. Heureux l’homme qui m’écoute, et qui veille tous les jours à ma porte, et qui se tient à la porte de ma maison. Celui qui me trouvera, trouvera la vie, et puisera le salut dans le Seigneur.
\textbf{\rb\ Deo grátias.}

\subsection*{Antiennes}
Vous êtes bénie et digne de vénération, Vierge Marie, qui avez été mère du Sauveur, sans que votre pureté ait subi d’atteinte.
\vb Vierge, Mère de Dieu, Celui que tout l’univers ne peut contenir, s’est enfermé dans votre sein en se faisant homme.


Allelúia, allelúia. \vb\ Vous êtes heureuse, sainte Vierge Marie, et tout à fait digne de louange, car de vous est sorti le soleil de justice, le Christ notre Dieu. Alléluia.

\subsection*{Début du Saint Evangile selon saint Mathieu.}
\rubrica{Matth, 1, 1-16.}
Généalogie de Jésus-Christ, fils de David, fils d’Abraham. Abraham engendra Isaac ; Isaac engendra Jacob ; Jacob engendra Juda et ses frères ; Juda, de Thamar, engendra Pharès et Zara ; Phares engendra Esrom ; Esrom engendra Aram ; Aram engendra Aminadab ; Aminadab engendra Naasson ; Naasson engendra Salmon ; Salmon, de Rahab, engendra Booz ; Booz, de Ruth, engendra Obed ; Obed engendra Jessé ; Jessé engendra le roi David. David engendra Salomon de la femme d’Urie ; Salomon engendra Roboam ; Roboam engendra Abia ; Abia engendra Asa ; Asa engendra Josaphat ; Josaphat engendra Joram ; Joram engendra Ozias ; Ozias engendra Joatham ; Joatham engendra Achaz ; Achaz engendra Ezéchias ; Ezéchias engendra Manassé ; Manassé engendra Amon ; Amon engendra Josias ; Josias engendra Jéchonias et ses frères, au temps de la déportation à Babylone. Après la déportation à Babylone, Jéchonias engendra Salathiel ; Salatheil engendra Zorobabel ; Zorobabel engendra Abioud ; Abioud engendra Eliacim ; Eliacim engendra Azor ; Azor engendra Sadoc ; Sadoc engendra Achim ; Achim engendra Elioud ; Elioud engendra Eléazar ; Eléazar engendra Matthan ; Matthan engendra Jacob ; Jacob engendra Joseph l’époux de Marie, de laquelle est né Jésus, qu’on appelle Christ.
{\textbf \rb\ Laus tibi Christe.}

\rubrica{Chant du Credo page \pageref{Credo}}

\subsection*{Antienne d'offertoire}
Vous êtes bienheureuse, Vierge Marie, qui avez porté le Créateur de toutes choses ; vous avez enfanté celui qui vous a créée, et vous demeurez à jamais Vierge.

\subsection*{Oraison Secrète}
Qu’elle nous porte secours, Seigneur, la bonté de votre Fils unique, qui né d’une Vierge, n’a point altéré l’intégrité de sa mere mais l’a consacrée, afin que nous purifiant de nos fautes en la solennité de sa Nativité, il vous rende notre oblation agréable, lui Jésus-Christ Notre-Seigneur.

\subsection*{Chant de communion}
Bienheureux le sein de la Vierge Marie, qui a porté le Fils du Père éternel.

\subsection*{Postcommunion}
Nous avons reçu, Seigneur, les choses saintes qui vous sont offertes en cette solennité annuelle, faites, nous vous en supplions, qu’elles nous donnent les remèdes spirituels utiles à la vie temporelle et conduisant à la vie éternelle.
{\textbf \rb\ Amen.}
\end{multicols}