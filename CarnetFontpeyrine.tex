% Afficher des recommendations concernant la syntaxe.
\RequirePackage[orthodox,l2tabu]{nag}
\RequirePackage{luatex85}
% Paramètres du document.
\documentclass[%
a5paper%                       Taille de page.
,11pt%                         Taille de police.
,DIV=15%                       Plus grand => des marges plus petites.
,titlepage=on%                 Faut-il une page de titre ?
,headings=optiontoheadandtoc%  Effet des paramètres optionnels de section.
,headings=small%
,parskip=false%
,openany%
]{scrbook}
\renewcommand*\partheademptypage{\thispagestyle{empty}}
\newcounter{facteur}\setcounter{facteur}{17}%%%%%%%%%%%%%% Paramètre pour la taille globale des partitions. par défaut~: 17
%\usepackage{geometry}
\usepackage{gredoc,mudoc,lyluatex}
\usepackage{pdfpages,transparent,array,ltablex}

%%%%%%%%%%%%%%%%%%%%%%% Paramètres variables %%%%%%%%%%%%%%%%%%%%%%%%%%%%%%%%%%%%%%%%%%%%%%%%%%
%%% Taille des partitions grégoriennes.                                                      %%
%\grechangedim{overhepisemalowshift}{.7mm}{scalable}
%\grechangedim{hepisemamiddleshift}{1.4mm}{scalable}
%\grechangedim{overhepisemahighshift}{2.1mm}{scalable}
%\grechangedim{vepisemahighshift}{2.1mm}{scalable}
%\grechangestafflinethickness{50} %%% epaisseur des lignes
\grechangestaffsize{\value{facteur}}%%%%% 
%%%%%%%%%%%%%%%%%%%%%%%%%%%%%%%%%%%%%%%%%%%%%%%%%%%%%%%%%%%%%%%%%%%%%%%%%%%%%%%%%%%%%%%%%%%%%%%
% Par souci de clarté, la définition des commandes est reportée dans un document annexe.

%\addtolength{\voffset}{2mm}\addtolength{\headsep}{-2mm}
%\setlength{\extrarowheight}{2mm}

\addto\captionsfrench{%
  \renewcommand{\indexname}{Index des chants}%
}

\pdfcompresslevel=9

\newcommand{\lieu}[1]{\hfill\linebreak[3]\hspace*{\stretch{1}}\nolinebreak\mbox{\emph{(#1)}}}

\newcommand{\commandement}[1]{\noindent\textbf{#1}}

\newcommand{\schola}[1]{}\newcommand{\foule}[1]{#1}
\providecommand{\dest}{foule}

\newcommand{\bgimage}[1]{%
\raisebox{-.45\paperheight}[0pt][0pt]{%
  \transparent{0.3}%
  \includegraphics[width=.7\paperwidth,height=.7\paperheight,keepaspectratio=true]{img/#1}%
  }%
}

\def\arraystretch{1.2}

\newcommand{\reponsegras}[2]{
\versio{\textbf{#1}}{{#2}}
}

\title{Jubilé de Notre-Dame de Fontpeyrine}
\date{}

\let\oldaddchap\addchap
\def\addchap#1{\oldaddchap{#1}\markright{Pèlerinage du Christ-Roi}}

\def\blindsection#1{\markright{#1}\addcontentsline{toc}{section}{#1}}
%%%%%%%%%%%%%%%%%%%%%%%%%%%%%%%%%%%%%%%%%%%%%%%%%%%%%%%%%%%%%%%%%%%%%%%%%%%%%%%%
%%%%%%%%%%%%%%%%%%%%%% Début du document %%%%%%%%%%%%%%%%%%%%%%%%%%%%%%%%%%%%%%%
%%%%%%%%%%%%%%%%%%%%%%%%%%%%%%%%%%%%%%%%%%%%%%%%%%%%%%%%%%%%%%%%%%%%%%%%%%%%%%%%
\begin{document}

%\maketitle
{\pagestyle{empty}
\foule{\includepdf{img/Couverture}}
\schola{\includepdf{Couverture-schola}}
\clearpage}

\vspace*{\stretch{1}}

{
\def\arraystretch{1}
{\centering\Large\textbf{Programme du Jubilé}\par}

\medskip\thispagestyle{empty}

\begin{tabularx}{\textwidth-\parindent}{l!{:}X}
\multicolumn{2}{l}{\textbf{Dimanche 2 juillet}}\\
9h00	& Procession.
	\lieu{Départ de l'église de Tursac}\\
10h30	& Messe par son Excellence Mgr de Galarreta.
	\lieu{Sanctuaire}\\
12h00	& Repas tiré du sac\\
12h00	& Ouverture des stands\\
15h00	& Bénédiction solennelle de la Croix du Jubilé.\\
15h00	& Traditionelle procession du chapelet\\
17h00   & Clôture de la journée\\
\end{tabularx}
}
\vspace*{\stretch{1}}
\part[Chants de procession]{Chants de procession\\
\medskip
\includegraphics[width=.5\textwidth]{img/ProcessionSeule.png}
}

Abb-Jean-Robin-1920-2002-Kyrie
\lilypondfile[staffsize=19]{ly/Abb-Jean-Robin-1920-2002-Kyrie.ly}

\titre{Les saints et les anges}
\chanson[position=2col,numero=1]{ly/AveMariaDeLourdes/LesSaintsEtLesAnges}

\titre[espace=15]{Nous voulons Dieu}
\chanson[position=2col,numero=1]{ly/NousVoulonsDieu/NousVoulonsDieu}

\titre[espace=15]{J'irai la voir un jour}
\chanson[position=2col,numero=1]{ly/JIraiLaVoirUnJour/JIraiLaVoirUnJour}

\titre[espace=15]{Tandis que le monde proclame}
\chanson[position=2col,numero=1]{ly/ParleCommandeRegne/ParleCommandeRegne}

\titre[espace=15]{Je suis chrétien}
\chanson[position=2col,numero=1]{ly/JeSuisChretien/JeSuisChretien}

\titre[espace=15]{Vierge sainte}
\chanson[position=2col,numero=1]{ly/ViergeSainte/ViergeSainte}

\titre[espace=15]{Laudate Mariam}
\chanson[position=2col,numero=1]{ly/LaudateMariam/LaudateMariam}

\titre[espace=15]{J'irai la voir un jour}
\chanson[position=2col,numero=1]{ly/JIraiLaVoirUnJour/JIraiLaVoirUnJour}

\titre[espace=15]{Ave Maris Stella}
\chanson[position=2col,numero=1]{ly/AveMarisStella/AveMarisStella}

\titre[espace=15]{Reine de France}
\chanson[numero=1]{ly/ReineDeFrance/ReineDeFrance}

\titre[espace=15]{Je mets ma confiance}
\chanson[position=2col,numero=1]{ly/JeMetsMaConfiance/JeMetsMaConfiance}

\titre[espace=15]{J'irai la voir un jour}
\chanson[position=2col,numero=1]{ly/JIraiLaVoirUnJour/JIraiLaVoirUnJour}

\titre[espace=15]{J'irai la voir un jour}
\chanson[position=2col,numero=1]{ly/JIraiLaVoirUnJour/JIraiLaVoirUnJour}

\titre[espace=15]{Vierge de Fontpeyrine}
\lilypondfile[staffsize=19]{ly/ViergeDeFontpeyrine/ViergeDeFontpeyrine.ly}
\chanson[position=2col,numero=2,refrain=non]{ly/ViergeDeFontpeyrine/ViergeDeFontpeyrine}
%\lilypondfile[staffsize=15]{ly/ViergeDeFontpeyrine/ViergeDeFontpeyrine-dernier.ly}

\titre{C'est le Périgord}
\lilypondfile[staffsize=15]{ly/CestLePerigord/CestLePerigord.ly}
\chanson[position=2col,numero=2]{ly/CestLePerigord/CestLePerigord}

\titre{Litanies de la très sainte Vierge}

\part[Messe de la Visitation]{Messe de la Visitation\\
\bigskip
\includegraphics[width=\textwidth]{img/VisitationGiotto.jpg}

\rubrica{ Le 2 juillet 1769, fête de la Visitation, un orage, tel qu'on n'en avait pas vu depuis longtemps, dévasta les campagnes de Saint-Cyprien, du Bugue, et de Terrasson. La paroisse de Tursac et le sanctuaire de Fontpeyrine furent seuls épargnés.
C'est pourquoi, les représentants de la ville de Tursac et sa pieuse population firent vœu d'aller au sanctuaire tous les ans en procession, au jour anniversaire du bienfait, pour remercier Notre-Dame de Fontpeyrine.}}
\subsection*{Introït}
\rubrica{Pendant que le l'officiant se prépare à monter à l'autel avec les \emph{prières au bas de l'autel}, nous chantons l'Introït}
\versio{Salve, sancta Parens, eníxa puérpera Regem~: qui cælum terrámque regit in sǽcula sæculórum.	}
{Salut, ô Mère sainte~; mère qui avez enfanté le Roi qui régit le ciel et la terre dans les siècles des siècles.}
\versio{\vb\ Eructávit cor meum verbum bonum~: dico ego ópera mea Regi. \vb\ Glória Patri.}{De mon cœur a jailli une parole excellente, c’est que je consacre mes œuvres à mon Roi. \vb\ Gloire au Père.}
\rubrica{On chante ensuite le \emph{Kyrie} et le \emph{Gloria}}

\subsection*{Collecte}
\versio{%
\vb\ Dóminus vobíscum.}{%
\vb\ Le Seigneur soit avec vous.}

\reponsegras{%
\rb\ Et cum spíritu tuo.}{%
\rb\ Et avec votre esprit.}

\versio{Orémus.\\
Fámulis tuis, quǽsumus, Dómine, cæléstis grátiæ munus impertíre~: ut, quibus beátæ Vírginis partus éxstitit salútis exórdium~; Visitatiónis eius votiva sollémnitas, pacis tríbuat increméntum. Per Dóminum.}
{Prions\\
Seigneur, nous vous prions d’accorder à vos serviteurs le don de la grâce céleste~: et, comme l’enfantement de la bienheureuse Vierge a été le principe de leur salut~; qu’ainsi la pieuse solennité de sa Visitation leur procure un accroissement de paix. Par Notre Seigneur.}
\reponsegras{%
\rb\ Amen.}{%
\rb\ Ainsi soit-il.}

\subsection*{Lecture du Livre de la Sagesse.}
\rubrica{Cant. 2, 8-14.}
Voici qu’il vient, bondissant sur les montagnes, sautant sur les collines. Mon bien-aimé est semblable à la gazelle, ou au faon des biches. Le voici, il est derrière notre mur, regardant par la fenêtre, épiant par le treillis. Voici, mon bien-aimé me dit~: "Lève-toi, hâte-toi, mon amie, ma colombe, ma belle, et viens ! Car voici que l’hiver est fini~; la pluie a cessé, elle a disparu. Les fleurs ont paru sur notre terre, le temps des chants est arrivé~; la voix de la tourterelle s’est fait entendre dans nos campagnes~; le figuier pousse ses fruits naissants, les vignes en fleur donnent son parfum. Lève-toi, mon amie, ma belle, et viens ! Ma colombe, qui te tiens dans la fente du rocher, dans l’abri des parois escarpées, montre-moi ton visage, que ta voix résonne à mes oreilles~; car ta voix est douce, et ton visage charmant.
\textbf{\rb\ Deo grátias.}

\subsection*{Lecture du Saint Évangile selon saint Luc.}
\rubrica{Luc. 1, 39-47.}
En ces jours-là~: Marie partit et s’en alla en hâte vers la montagne, en une ville de Juda. Et elle entra dans la maison de Zacharie, et salua Élisabeth. Or, quand Élisabeth entendit la salutation de Marie, l’enfant tressaillit dans son sein, et elle fut remplie du Saint-Esprit. Et elle s’écria à haute voix, disant~: "Vous êtes bénie entre les femmes, et le fruit de vos entrailles est béni. Et d’où m’est-il donné que la mère de mon Seigneur vienne à moi ? Car votre voix, lorsque vous m’avez saluée, n’a pas plus tôt frappé mes oreilles, que l’enfant a tressailli de joie dans mon sein. Heureuse celle qui a cru ! Car elles seront accomplies les choses qui lui ont été dites de la part du Seigneur !" Et Marie dit~: "Mon âme glorifie le Seigneur, et mon esprit tressaille de joie en Dieu, mon Sauveur".
\reponsegras{%
\rb\ Laus tibi Christe.}{%
\rb\ Christ, louange à vous.}

\subsection*{Offertoire}
\rubrica{À l'Offertoire l'Église offre le Corps et le Sang du Christ, représentés par le pain et le vin. Notre Seigneur a communiqué à son Église le pouvoir d'offrir le même sacrifice qu'il offrit sur la Croix. L'Église s'y unit comme victime. Nous sommes donc une seule victime avec le Christ, unissant nos propres offrandes et nos sacrifices à celui du Christ et de l'Église. Notre participation à cette offrande est exprimée par le chant de l'offertoire.}
Recevez, Père saint, Dieu éternel et tout-puissant, cette offrande sans tache, que moi, votre indigne serviteur, je vous présente, à vous, mon Dieu vivant et vrai, pour mes péchés, offenses et négligences sans nombre, pour tous ceux qui m'entourent, ainsi que pour tous les fidèles vivants et morts~: qu'elle serve à mon salut et au leur pour la vie éternelle. Ainsi soit-il.
Dieu, ✠ qui, d'une manière admirable, avez créé la nature humaine dans sa noblesse, et l'avez restaurée d'une manière plus admirable encore, accordez-nous, selon le mystère de cette eau et de ce vin, de prendre part à la divinité de celui qui a daigné partager notre humanité, Jésus-Christ votre Fils, Notre Seigneur, qui, étant Dieu, vit et règne avec vous en l'unité du Saint-Esprit, dans tous les siècles des siècles. \mbox{Ainsi soit-il}.\looseness=1

\subsection*{Consécration}
\rubrica{Le prêtre récite alors l'histoire de l'institution de l'Eucharistie en accomplissant les mêmes gestes que le Christ.}

\versio{%
Qui prídie quam paterétur, accépit panem in sanctas ac venerábiles manus suas, et elevátis óculis in cælum ad te Deum Patrem suum omnipoténtem, tibi grátias agens, bene ✠ díxit, fregit, dedítque discípulis suis, dicens~: Accípite, et manducáte ex hoc omnes.}{%
Celui-ci, la veille de sa Passion, prit du pain dans ses mains saintes et adorables, et, les yeux levés au ciel vers vous, Dieu, son Père tout-puissant, vous rendant grâces, il ✠ bénit ce pain, le rompit et le donna à ses disciples en disant~: Prenez et mangez-en tous.}


\rubrica{Le prêtre prononce alors les paroles mêmes de Notre-Seigneur. Par ces paroles il opère la conversion du pain au saint Corps du Christ.}

\versio{%
\scalebox{.91}[1]{\textsc{\addfontfeatures{Renderer=Basic}Hoc est enim Corpus meum.}}\par%
}{%
\textsc{\addfontfeatures{Renderer=Basic}Car ceci est mon Corps.}\par%
}

%\vspace{1\baselineskip plus 2\baselineskip}

\rubrica{Ces paroles étant prononcées, le prêtre fait la génuflexion pour adorer le saint Corps, l'élève pour le présenter à l'adoration des fidèles, puis reprend le récit de l'institution de l'eucharistie~:}

\versio{%
Símili modo postquam cenátum est, accípiens et hunc præclárum Cálicem in sanctas ac venerábiles manus suas~: item tibi grátias agens, bene ✠ díxit, dedítque discípulis suis, dicens~: Accípite, et bíbite ex eo omnes.}{%
De même, après le repas, il prit aussi ce précieux calice dans ses mains saintes et adorables, vous rendit grâces encore, le ✠ bénit et le donna à ses disciples en disant~: Prenez et buvez-en tous.}

\rubrica{Prononçant alors les paroles mêmes de Notre-Seigneur, le prêtre opère la conversion du vin au précieux Sang du Christ.}
%
\nopagebreak\smallskip%
%
\versio{%
\textsc{\addfontfeatures{Renderer=Basic}Hic est enim Calix Sánguinis mei, novi et ætérni Testaménti~: mystérium fídei~: qui pro vobis et pro multis effundétur in remissiónem peccatórum.}%
}{%
\textsc{\addfontfeatures{Renderer=Basic}Car ceci est le Calice de mon Sang, le Sang de l'Alliance nouvelle et éternelle, le mystère de la foi, qui sera versé pour vous et pour un grand nombre en rémission des péchés.}%
}

\smallskip

\versio{%
Hæc quotiescúmque fecéritis, in mei memóriam faciétis.}{%
Toutes les fois que vous ferez cela, vous le ferez en mémoire de moi.}%

%\vspace{1\baselineskip plus 2\baselineskip}

\rubrica{Ces paroles étant prononcées, le prêtre fait la génuflexion pour adorer le précieux Sang. Il l'élève pour le présenter à l'adoration, et fait de nouveau la génuflexion.}

\subsection*{Communion}
Seigneur Jésus-Christ, qui avez dit à vos Apôtres~: C'est la paix que je vous laisse en héritage, c'est ma paix je vous donne, ne regardez pas mes péchés, mais la foi de votre Église~; daignez, selon votre volonté, lui donner la paix et la rassembler dans l'unité, vous qui, étant Dieu, vivez et régnez dans tous les siècles des siècles. Ainsi soit-il.
\rubrica{Le baiser de paix, manifestant l'union dans la paix du Christ, est hiérarchiquement transmis de l'autel jusqu'au dernier degré du clergé. \emph{Ne pense pas que ce baiser soit comme ceux qui se donnent sur la place entre amis ordinaires. Il unit les âmes entre elles~; il est réconciliation, et pour cette raison il est saint.} (S.~Cyrille de Jérusalem)}
Seigneur Jésus-Christ, Fils du Dieu vivant, qui, accomplissant la volonté du Père dans une œuvre commune avec le Saint-Esprit, avez par votre mort donné la vie au monde, délivrez-moi par votre Corps et votre Sang infiniment saints de tous mes péchés et de tout mal. Faites que je reste toujours attaché à vos commandements, et ne permettez pas que je sois jamais séparé de vous, qui, étant Dieu, vivez et régnez avec Dieu le Père et le Saint-Esprit dans les siècles des siècles. Ainsi soit-il.
Seigneur Jésus-Christ, si j'ose recevoir votre Corps malgré mon indignité, que cela n'entraîne pour moi ni jugement ni condamnation, mais, par votre miséricorde, me serve de sauvegarde et de remède pour l'âme et pour le corps, vous qui, étant Dieu, vivez et régnez avec Dieu le Père en l'unité du Saint-Esprit, dans tous les siècles des siècles. Ainsi soit-il.

\rubrica{Après avoir communié, le prêtre se retourne vers les fidèles en leur présentant le saint Corps du Christ.}

\versio{%
Ecce Agnus Dei, ecce qui tollis peccáta mundi.}{%
Voici l'Agneau de Dieu, voici celui qui enlève les péchés du monde.}

\rubrica{Nous répondons par les paroles du bienheureux centurion. Cette fois ce n'est plus la guérison d'un serviteur que nous demandons, mais celle de notre âme.}
%
\versio{%
Dómine, non sum dignus, ut intres sub tectum meum~: sed tantum dic verbo, et sanábitur ánima mea. \emph{ter}}{%
Seigneur, je ne suis pas digne que vous entriez sous mon toit~; mais dites seulement une parole, et mon âme sera guérie. \emph{ter}}

\rubrica{Aux premiers siècles le diacre invitait ceux qui ne pouvaient pas communier à se retirer. Aujourd'hui l'Église est plus indulgente mais les conditions pour communier sont les mêmes~: \textbf{être baptisé et catholique~; ne pas avoir de péché mortel sur la conscience, ce qui suppose notamment l'observation des lois de l'Église au sujet du mariage~; avoir observé le jeûne eucharistique.} La discipline actuelle est d'une heure de jeûne avant la communion. Nous conseillons cependant de s'en tenir à la discipline d'avant le Concile Vatican II~: trois heures pour la nourriture solide, une heure pour le liquide non alcoolisé.\\
La sainte communion est reçue à genoux et directement dans la bouche.}

\needspace{3\baselineskip}
\rubrica{En donnant la sainte Eucharistie, le prêtre dit~:}

\versio{%
Corpus Dómini nostri Iesu Christi custódiat ánimam tuam in vitam ætérnam. Amen.}{%
Que le Corps de Notre Seigneur Jésus-Christ garde votre âme pour la vie éternelle. Ainsi soit-il.}

\rubrica{%
On ne répond rien.}

\rubrica{%
Jésus-Christ est le Bon Pasteur qui donne sa vie pour ses brebis. \emph{Je suis venu pour qu'ils aient la vie et qu'ils l'aient en abondance}. Il nous communique cette vie dans la sainte Eucharistie.%
}

\subsection*{Chant de communion}

Bienheureux le sein de la Vierge Marie, qui a porté le Fils du Père éternel.

\subsection*{Postcommunion}
\versio{%
\vb\ Dóminus vobíscum.}{%
\vb\ Le Seigneur soit avec vous.}

\reponsegras{%
\rb\ Et cum spíritu tuo.}{%
\rb\ Et avec votre esprit.}
\versio{Súmpsimus, Dómine, celebritátis ánnuæ votiva sacraménta~: præsta, quǽsumus~; ut et temporális vitæ nobis remédia prǽbeant et ætérnæ. Per Dóminum.}
{Nous avons reçu, Seigneur, les choses saintes qui vous sont offertes en cette solennité annuelle, faites, nous vous en supplions, qu’elles nous donnent les remèdes spirituels utiles à la vie temporelle et conduisant à la vie éternelle.}

\reponsegras{%
\rb\ Amen.}{%
\rb\ Ainsi soit-il.}

\subsection*{Prière d'action de grâces\\ Prière de Fontpeyrine}

{\centering {\textit{Notre-Dame de Fontpeyrine,\\
qui depuis des
siècles 
accordez de nombreuses faveurs\\
à ceux qui ont recours à votre puissante intercession,\\
obtenez, nous vous en supplions,\\
à nous vos humbles serviteurs, \\
en souvenir de votre bienheureuse Nativité,\\
ce complément de grâce\\
que nous implorons à genoux devant vous.\\
Nous l’attendons avec confiance,\\
malgré notre indignité, ô Mère du Sauveur,\\
de votre maternelle bonté\\
et de votre bienveillante protection.\\
Ainsi soit-il.\\}}}

\clearpage
\addcontentsline{toc}{part}{Bénédiction solennelle de la Croix du Jubilé}
{\subsection*{Bénédiction solennelle de la Croix du Jubilé}}
\versio{%
\vb\ \crux Adiutórium nostrum in nómine Dómini.}{%
\vb\ \crux Notre secours est dans le nom du Seigneur.}

\reponsegras{%
\rb\ Qui fecit cælum et terram.}{%
\rb\ Qui a fait le ciel et la terre.}

\versio{%
\vb\ Dómine, exáudi oratiónem meam.}{%
\vb\ Seigneur, exaucez ma prière.}

\reponsegras{%
\rb\ Et clamor meus ad te véniat.}{%
\rb\ Et que mon cri s'élève jusqu'à vous.}

\versio{%
\vb\ Dóminus vobíscum.}{%
\vb\ Le Seigneur soit avec vous.}

\reponsegras{%
\rb\ Et cum spíritu tuo.}{%
\rb\ Et avec votre esprit.}
\versio{Orémus.
Rogámus te, Dómine sancte. Pater omnípotens, ætérne Deus: ut dignéris bene ✠ dícere hoc
signum Crucis, ut sit remédium salutáre géneri humáno; sit solíditas fídei, proféctus bonórum
óperum, redémptio animárum; sit solámen, et protéctio, ac tutéla contra sæva jácula inimicórum.
Per Christum Dóminum nostrum.  \textbf{\rb\ Amen}.}
{Nous vous prions, Seigneur saint, Père tout-puissant~: que vous daigniez bénir ✠ ce signe de votre Croix afin qu'il soit un remède pour le salut du genre humain~; qu'il soit la solidité de la foi, la profusion des bonnes œuvres, la rédemption de notre âme; qu'il soit le compagnon et le protecteur et le refuge contre les sévères assauts de l'ennemi. Par le Christ Notre Seigneur. \rb\ Amen.}
\versio{Orémus,
Béne ✠ dic, Dómine Jesu Christe, hanc Crucem, per quam eripuísti mundum a potestáte daémonum,
et superásti passióne tua suggestórem peccáti, qui gaudébat in prævaricatióne primi hóminis
per ligni vétiti sumptiónem.}
{Bénissez ✠ , Seigneur Jésus Christ,
cette croix par
laquelle vous avez arraché le monde au pouvoir des démons, et terrassé lors de
votre passion l'instigateur du
péché, qui se réjouissait de
ce que le premier homme fût
tombé en mangeant du fruit défendu.
}
\versio{\rubrica{Hic aspergatur aqua benedicta.}}{\rubrica{Ici, le prêtre asperge la croix d'eau bénite.}}
\versio{Sanctificétur hoc signum Crucis in nómine Pa ✠ tris, et Fí ✠ lii, et Spíritus ✠ Sancti; ut orántes, inclinantésque se propter Dóminum
ante istam Crucem, invéniant córporis et ánimæ sanitátem. Per eúmdem Christum Dóminum
nostrum. \textbf{\rb\ Amen.}}
{Que cette effigie de la
Croix soit sanctifiée au nom
du ✠ Père, et du ✠ Fils, et du Saint ✠ Esprit, en sorte que tous ceux qui prieront
et s'inclineront en l'honneur
du Seigneur devant cette
croix obtiennent la santé de
l'âme et du corps. Par le même Christ, notre Seigneur. \rb\ Amen}

\part[Procession du chapelet]{Procession du chapelet\\
%\large{9h00}\\
\medskip
\includegraphics[width=.7\textwidth]{img/Monument.png}
}
%\part{Procession du chapelet\\
%\includegraphics[width=.3\textwidth]{img/Monument.png}
%}

\titre{Chez nous soyez Reine}
\chanson[position=2col,numero=1]{ly/ChezNousSoyezReine/ChezNousSoyezReine}

\titre{J'irai la voir un jour}
\chanson[position=2col,numero=1]{ly/JIraiLaVoirUnJour/JIraiLaVoirUnJour}

\titre{Je suis chrétien}
\chanson[position=2col,numero=1]{ly/JeSuisChretien/JeSuisChretien}

\newpage
\vspace*{\stretch{1}}
{\centering{Si vous voulez nous aider, vous pouvez adresser vos dons à :

\textsc{Association Notre-Dame de Fontpeyrine}


\textit{adresse administrative :}\\
5 rue de Clairat,\\
24100 BERGERAC\\


Ordre des chèques : « Notre-Dame de Fontpeyrine »\\
Ou par virement :\\
Numéro de compte : 15589 24581 06442912040 66\\
IBAN : FR76 1558 9245 8106 4429 1204 066\\}
}
\vspace*{\stretch{1}}

\tableofcontents
\printindex
\newpage

%\vfill\thispagestyle{empty}

{\thispagestyle{empty}\centering
\vspace*{\stretch{2}}

\includegraphics[width=.6\textwidth]{img/SceauJubilé.pdf}

\vspace*{\stretch{3}}

\footnotesize Association N-D. de Fontpeyrine\\
Aumônerie assurée par la Fraternité sacerdotale Saint-Pie-X\par}
\end{document}